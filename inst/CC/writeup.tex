\documentclass[11pt]{article}
\usepackage{asa}
\usepackage{xspace}
\usepackage{natbib}
\usepackage{url}
\usepackage{afterpage}
\usepackage{pdflscape}
\usepackage{geometry} % see geometry.pdf on how to lay out the page. There's lots.
\geometry{letterpaper} % or letter or a5paper or ... etc
% \geometry{landscape} % rotated page geometry
% See the ``Article customise'' template for come common customisations

%\definecolor{Blue}{rgb}{0,0,0.5}
\usepackage{color}
\definecolor{purple}{rgb}{.4,0,.8}
\newcommand{\hh}[1]{{\color{magenta} #1}}
\newcommand{\sct}[1]{{\color{purple} #1}}
\newcommand{\sctq}[1]{{\color{red} #1}} %things in red are questions for you or concerns that i have.

\title{A Geometry for Network Visualization in \MakeLowercase{\texttt{ggplot2}}}
\author{Sam Tyner and Heike Hofmann}
\date{} % delete this line to display the current date
%%% BEGIN DOCUMENT
\usepackage{Sweave}
\begin{document}

\maketitle
\begin{abstract}
There are many implementations of static network visualization in R, but none are equipped with the flexibility and functionality of \texttt{ggplot2}.  \texttt{geom\_net} was created to fill this gap. Using two data frames to describe vertex and edge information, \texttt{geom\_net} makes use of the underlying structure of \texttt{ggplot2} to visualize networks. This makes it possible to easily facet networks according to covariates  or change aesthetics such as shape, size, or color according to additional edge or vertex information.
\end{abstract}
